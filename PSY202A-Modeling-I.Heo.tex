% Options for packages loaded elsewhere
\PassOptionsToPackage{unicode}{hyperref}
\PassOptionsToPackage{hyphens}{url}
%
\documentclass[
]{book}
\usepackage{amsmath,amssymb}
\usepackage{iftex}
\ifPDFTeX
  \usepackage[T1]{fontenc}
  \usepackage[utf8]{inputenc}
  \usepackage{textcomp} % provide euro and other symbols
\else % if luatex or xetex
  \usepackage{unicode-math} % this also loads fontspec
  \defaultfontfeatures{Scale=MatchLowercase}
  \defaultfontfeatures[\rmfamily]{Ligatures=TeX,Scale=1}
\fi
\usepackage{lmodern}
\ifPDFTeX\else
  % xetex/luatex font selection
\fi
% Use upquote if available, for straight quotes in verbatim environments
\IfFileExists{upquote.sty}{\usepackage{upquote}}{}
\IfFileExists{microtype.sty}{% use microtype if available
  \usepackage[]{microtype}
  \UseMicrotypeSet[protrusion]{basicmath} % disable protrusion for tt fonts
}{}
\makeatletter
\@ifundefined{KOMAClassName}{% if non-KOMA class
  \IfFileExists{parskip.sty}{%
    \usepackage{parskip}
  }{% else
    \setlength{\parindent}{0pt}
    \setlength{\parskip}{6pt plus 2pt minus 1pt}}
}{% if KOMA class
  \KOMAoptions{parskip=half}}
\makeatother
\usepackage{xcolor}
\usepackage{color}
\usepackage{fancyvrb}
\newcommand{\VerbBar}{|}
\newcommand{\VERB}{\Verb[commandchars=\\\{\}]}
\DefineVerbatimEnvironment{Highlighting}{Verbatim}{commandchars=\\\{\}}
% Add ',fontsize=\small' for more characters per line
\usepackage{framed}
\definecolor{shadecolor}{RGB}{248,248,248}
\newenvironment{Shaded}{\begin{snugshade}}{\end{snugshade}}
\newcommand{\AlertTok}[1]{\textcolor[rgb]{0.94,0.16,0.16}{#1}}
\newcommand{\AnnotationTok}[1]{\textcolor[rgb]{0.56,0.35,0.01}{\textbf{\textit{#1}}}}
\newcommand{\AttributeTok}[1]{\textcolor[rgb]{0.13,0.29,0.53}{#1}}
\newcommand{\BaseNTok}[1]{\textcolor[rgb]{0.00,0.00,0.81}{#1}}
\newcommand{\BuiltInTok}[1]{#1}
\newcommand{\CharTok}[1]{\textcolor[rgb]{0.31,0.60,0.02}{#1}}
\newcommand{\CommentTok}[1]{\textcolor[rgb]{0.56,0.35,0.01}{\textit{#1}}}
\newcommand{\CommentVarTok}[1]{\textcolor[rgb]{0.56,0.35,0.01}{\textbf{\textit{#1}}}}
\newcommand{\ConstantTok}[1]{\textcolor[rgb]{0.56,0.35,0.01}{#1}}
\newcommand{\ControlFlowTok}[1]{\textcolor[rgb]{0.13,0.29,0.53}{\textbf{#1}}}
\newcommand{\DataTypeTok}[1]{\textcolor[rgb]{0.13,0.29,0.53}{#1}}
\newcommand{\DecValTok}[1]{\textcolor[rgb]{0.00,0.00,0.81}{#1}}
\newcommand{\DocumentationTok}[1]{\textcolor[rgb]{0.56,0.35,0.01}{\textbf{\textit{#1}}}}
\newcommand{\ErrorTok}[1]{\textcolor[rgb]{0.64,0.00,0.00}{\textbf{#1}}}
\newcommand{\ExtensionTok}[1]{#1}
\newcommand{\FloatTok}[1]{\textcolor[rgb]{0.00,0.00,0.81}{#1}}
\newcommand{\FunctionTok}[1]{\textcolor[rgb]{0.13,0.29,0.53}{\textbf{#1}}}
\newcommand{\ImportTok}[1]{#1}
\newcommand{\InformationTok}[1]{\textcolor[rgb]{0.56,0.35,0.01}{\textbf{\textit{#1}}}}
\newcommand{\KeywordTok}[1]{\textcolor[rgb]{0.13,0.29,0.53}{\textbf{#1}}}
\newcommand{\NormalTok}[1]{#1}
\newcommand{\OperatorTok}[1]{\textcolor[rgb]{0.81,0.36,0.00}{\textbf{#1}}}
\newcommand{\OtherTok}[1]{\textcolor[rgb]{0.56,0.35,0.01}{#1}}
\newcommand{\PreprocessorTok}[1]{\textcolor[rgb]{0.56,0.35,0.01}{\textit{#1}}}
\newcommand{\RegionMarkerTok}[1]{#1}
\newcommand{\SpecialCharTok}[1]{\textcolor[rgb]{0.81,0.36,0.00}{\textbf{#1}}}
\newcommand{\SpecialStringTok}[1]{\textcolor[rgb]{0.31,0.60,0.02}{#1}}
\newcommand{\StringTok}[1]{\textcolor[rgb]{0.31,0.60,0.02}{#1}}
\newcommand{\VariableTok}[1]{\textcolor[rgb]{0.00,0.00,0.00}{#1}}
\newcommand{\VerbatimStringTok}[1]{\textcolor[rgb]{0.31,0.60,0.02}{#1}}
\newcommand{\WarningTok}[1]{\textcolor[rgb]{0.56,0.35,0.01}{\textbf{\textit{#1}}}}
\usepackage{longtable,booktabs,array}
\usepackage{calc} % for calculating minipage widths
% Correct order of tables after \paragraph or \subparagraph
\usepackage{etoolbox}
\makeatletter
\patchcmd\longtable{\par}{\if@noskipsec\mbox{}\fi\par}{}{}
\makeatother
% Allow footnotes in longtable head/foot
\IfFileExists{footnotehyper.sty}{\usepackage{footnotehyper}}{\usepackage{footnote}}
\makesavenoteenv{longtable}
\usepackage{graphicx}
\makeatletter
\def\maxwidth{\ifdim\Gin@nat@width>\linewidth\linewidth\else\Gin@nat@width\fi}
\def\maxheight{\ifdim\Gin@nat@height>\textheight\textheight\else\Gin@nat@height\fi}
\makeatother
% Scale images if necessary, so that they will not overflow the page
% margins by default, and it is still possible to overwrite the defaults
% using explicit options in \includegraphics[width, height, ...]{}
\setkeys{Gin}{width=\maxwidth,height=\maxheight,keepaspectratio}
% Set default figure placement to htbp
\makeatletter
\def\fps@figure{htbp}
\makeatother
\setlength{\emergencystretch}{3em} % prevent overfull lines
\providecommand{\tightlist}{%
  \setlength{\itemsep}{0pt}\setlength{\parskip}{0pt}}
\setcounter{secnumdepth}{5}
\usepackage{booktabs}
\usepackage{amsthm}
\makeatletter
\def\thm@space@setup{%
  \thm@preskip=8pt plus 2pt minus 4pt
  \thm@postskip=\thm@preskip
}
\makeatother
\ifLuaTeX
  \usepackage{selnolig}  % disable illegal ligatures
\fi
\usepackage[]{natbib}
\bibliographystyle{apalike}
\usepackage{bookmark}
\IfFileExists{xurl.sty}{\usepackage{xurl}}{} % add URL line breaks if available
\urlstyle{same}
\hypersetup{
  pdftitle={{[}PSY202A{]} Statistical Modeling in Psychological Sciences},
  pdfauthor={Ihnwhi Heo, M.Sc.},
  hidelinks,
  pdfcreator={LaTeX via pandoc}}

\title{{[}PSY202A{]} Statistical Modeling in Psychological Sciences}
\author{\href{https://ihnwhiheo.github.io/}{Ihnwhi Heo, M.Sc.}}
\date{Fall 2024}

\begin{document}
\maketitle

{
\setcounter{tocdepth}{1}
\tableofcontents
}
\chapter{Introduction}\label{introduction}

Hi everyone! I'm Ihnwhi.

It is my great pleasure to be your guest lecturer for PSY202A.

Statistical modeling is a key component in conducting research in the psychological sciences. While many statistical toolkits are available to researchers, R is arguably one of the most useful free and open-source statistical software programs. It offers a dizzying array of analytic options to answer your important and exciting research questions. In the upcoming four lab sessions, you will be introduced to R and become familiar with its capabilities. These sessions are designed to help you get acquainted with the fundamentals of R and learn how to use it wisely as the next generation of psychologists. You will then be guided through summarizing and analyzing data using R.

Are you ready? Let's get it on!

\chapter{Introduction to R: Part 1}\label{introduction-to-r-part-1}

\section{What is it? Why called Mplus?}\label{what-is-it-why-called-mplus}

Mplus is a statistical modeling program that provides researchers with a flexible tool to analyze data

\begin{itemize}
\item
  Many models: regression, path analysis, factor analysis, SEM, MLM, longitudinal models, mixture model, mediation/moderation
\item
  Many data: cross-sectional, longitudinal, single-/multilevel, observed/latent, incomplete
\item
  Many variables: continuous, dichotomous, categorical, count
\item
  Many estimator: maximum likelihood, weighted least squares, Bayesian
\end{itemize}

\section{Syntax-based programming}\label{syntax-based-programming}

\begin{itemize}
\item
  Commands and subcommands (\url{https://www.statmodel.com/language.html})
\item
  Examples of commands? (\url{https://www.youtube.com/watch?v=XeRRtdmu23k})

  \begin{itemize}
  \tightlist
  \item
    We will be `mostly' using TITLE, DATA, VARIABLE, ANALYSIS, MODEL, OUTPUT commands
  \item
    But we will also be often using DEFINE, SAVEDATA, PLOT, MONTECARLO commands
  \end{itemize}
\end{itemize}

\section{Some tips when programming}\label{some-tips-when-programming}

\begin{enumerate}
\def\labelenumi{\arabic{enumi}.}
\item
  Comments can be added with exclamation marks (!)
\item
  Commands should end with colon (:), and subcommands should end with semicolon (;)
\item
  Syntax is not case sensitive
\item
  Data should consist of numeric values, with no variable names
\item
  Data and Mplus input file should be in the same directory (like an R working directory)
\end{enumerate}

\begin{itemize}
\tightlist
\item
  Otherwise, be sure to specify the correct directory
\end{itemize}

\section{Some tips about model command particularly}\label{some-tips-about-model-command-particularly}

\begin{enumerate}
\def\labelenumi{\arabic{enumi}.}
\item
  Start with a path diagram
\item
  Think of it as specifying model parameters
\item
  Care to the degrees of freedom (DF)
\end{enumerate}

\section{Example: Multiple linear regression using maximum likelihood estimation}\label{example-multiple-linear-regression-using-maximum-likelihood-estimation}

\subsection{Model syntax}\label{model-syntax}

\begin{Shaded}
\begin{Highlighting}[]
\SpecialCharTok{!}\NormalTok{ Title command}
\NormalTok{TITLE}\SpecialCharTok{:}\NormalTok{ Predicting album sales using ML multiple regression}

\SpecialCharTok{!}\NormalTok{ Data command}
\NormalTok{DATA}\SpecialCharTok{:}
    \SpecialCharTok{!}\NormalTok{ When data and input file are }\ControlFlowTok{in}\NormalTok{ the same working directory}
\NormalTok{    FILE IS Album Sales.csv; }\SpecialCharTok{!}\NormalTok{ Subcommands should end with ;}

    \SpecialCharTok{!}\NormalTok{ When data and input file are }\ControlFlowTok{in}\NormalTok{ the different working directory}
    \SpecialCharTok{!}\NormalTok{ FILE IS c}\SpecialCharTok{:}\NormalTok{\textbackslash{}desktop\textbackslash{}different folder\textbackslash{}Album Sales.csv;}

\SpecialCharTok{!}\NormalTok{ Variable command}
\NormalTok{VARIABLE}\SpecialCharTok{:}
    \SpecialCharTok{!}\NormalTok{ Column }\FunctionTok{names}\NormalTok{ (i.e., ALL variable names)}
\NormalTok{    NAMES ARE adverts sales airplay attract;}
    
    \SpecialCharTok{!}\NormalTok{ Variables that will be used }\ControlFlowTok{in}\NormalTok{ our analysis}
\NormalTok{    USEVARIABLES ARE adverts sales airplay;}
    
\SpecialCharTok{!}\NormalTok{ Analysis command}
\NormalTok{ANALYSIS}\SpecialCharTok{:}
\NormalTok{    ESTIMATOR IS ML; }\SpecialCharTok{!}\NormalTok{ This is the default}

\SpecialCharTok{!}\NormalTok{ Model command}
\NormalTok{MODEL}\SpecialCharTok{:}
    \SpecialCharTok{!}\NormalTok{ Let}\StringTok{\textquotesingle{}s predict sales using adverts and airplay}
\StringTok{    ! We regress sales on adverts and airplay}
\StringTok{    sales ON adverts airplay;}

\StringTok{    ! If you do not specify variances of and covariances between predictors}
\StringTok{    ! degrees of freedom (DF) are not correct}
\StringTok{    ! Variances of exogenous variable}
\StringTok{    adverts airplay;}
\StringTok{    ! Covariances between exogenous variable}
\StringTok{    adverts WITH airplay;}

\StringTok{! Output command}
\StringTok{OUTPUT:}
\StringTok{    TECH1 SAMPSTAT STDYX;}
\StringTok{    ! TECH1 to understand which parameters are being estimated}
\StringTok{    ! SAMPSTAT to check sample descriptive statistics}
\StringTok{    ! STDYX to standardize Y (i.e., DV) and X (i.e., IV)}
\end{Highlighting}
\end{Shaded}

\subsection{Part of the output file}\label{part-of-the-output-file}

\begin{Shaded}
\begin{Highlighting}[]
\NormalTok{MODEL RESULTS}

\NormalTok{                                                    Two}\SpecialCharTok{{-}}\NormalTok{Tailed}
\NormalTok{                    Estimate       S.E.  Est.}\SpecialCharTok{/}\NormalTok{S.E.    P}\SpecialCharTok{{-}}\NormalTok{Value}

\NormalTok{ SALES    ON}
\NormalTok{    ADVERTS            }\FloatTok{0.087}      \FloatTok{0.007}     \FloatTok{12.082}      \FloatTok{0.000}
\NormalTok{    AIRPLAY            }\FloatTok{3.589}      \FloatTok{0.285}     \FloatTok{12.608}      \FloatTok{0.000}

\NormalTok{ ADVERTS  WITH}
\NormalTok{    AIRPLAY          }\FloatTok{604.061}    \FloatTok{421.412}      \FloatTok{1.433}      \FloatTok{0.152}

\NormalTok{ Means}
\NormalTok{    ADVERTS          }\FloatTok{614.412}     \FloatTok{34.255}     \FloatTok{17.936}      \FloatTok{0.000}
\NormalTok{    AIRPLAY           }\FloatTok{27.500}      \FloatTok{0.865}     \FloatTok{31.777}      \FloatTok{0.000}
\end{Highlighting}
\end{Shaded}

\section{Additional materials}\label{additional-materials}

\begin{enumerate}
\def\labelenumi{\arabic{enumi}.}
\item
  Official website at \url{https://www.statmodel.com/}
\item
  User's guide and examples at \url{https://www.statmodel.com/ugexcerpts.shtml} \(\rightarrow\) Highly recommended!
\item
  Mplus YouTube channel at \url{https://www.youtube.com/c/MplusVideos}
\item
  QuantFish YouTube channel at \url{https://www.youtube.com/c/QuantFish}
\item
  Tutorials by Prof.~Rens van de Schoot and his students at \url{https://www.rensvandeschoot.com/tutorials/}
\end{enumerate}

\chapter{Introduction to R: Part 2}\label{introduction-to-r-part-2}

\section{What is it? Why called Mplus?}\label{what-is-it-why-called-mplus-1}

Mplus is a statistical modeling program that provides researchers with a flexible tool to analyze data

\begin{itemize}
\item
  Many models: regression, path analysis, factor analysis, SEM, MLM, longitudinal models, mixture model, mediation/moderation
\item
  Many data: cross-sectional, longitudinal, single-/multilevel, observed/latent, incomplete
\item
  Many variables: continuous, dichotomous, categorical, count
\item
  Many estimator: maximum likelihood, weighted least squares, Bayesian
\end{itemize}

\section{Syntax-based programming}\label{syntax-based-programming-1}

\begin{itemize}
\item
  Commands and subcommands (\url{https://www.statmodel.com/language.html})
\item
  Examples of commands? (\url{https://www.youtube.com/watch?v=XeRRtdmu23k})

  \begin{itemize}
  \tightlist
  \item
    We will be `mostly' using TITLE, DATA, VARIABLE, ANALYSIS, MODEL, OUTPUT commands
  \item
    But we will also be often using DEFINE, SAVEDATA, PLOT, MONTECARLO commands
  \end{itemize}
\end{itemize}

\section{Some tips when programming}\label{some-tips-when-programming-1}

\begin{enumerate}
\def\labelenumi{\arabic{enumi}.}
\item
  Comments can be added with exclamation marks (!)
\item
  Commands should end with colon (:), and subcommands should end with semicolon (;)
\item
  Syntax is not case sensitive
\item
  Data should consist of numeric values, with no variable names
\item
  Data and Mplus input file should be in the same directory (like an R working directory)
\end{enumerate}

\begin{itemize}
\tightlist
\item
  Otherwise, be sure to specify the correct directory
\end{itemize}

\section{Some tips about model command particularly}\label{some-tips-about-model-command-particularly-1}

\begin{enumerate}
\def\labelenumi{\arabic{enumi}.}
\item
  Start with a path diagram
\item
  Think of it as specifying model parameters
\item
  Care to the degrees of freedom (DF)
\end{enumerate}

\section{Example: Multiple linear regression using maximum likelihood estimation}\label{example-multiple-linear-regression-using-maximum-likelihood-estimation-1}

\subsection{Model syntax}\label{model-syntax-1}

\begin{Shaded}
\begin{Highlighting}[]
\SpecialCharTok{!}\NormalTok{ Title command}
\NormalTok{TITLE}\SpecialCharTok{:}\NormalTok{ Predicting album sales using ML multiple regression}

\SpecialCharTok{!}\NormalTok{ Data command}
\NormalTok{DATA}\SpecialCharTok{:}
    \SpecialCharTok{!}\NormalTok{ When data and input file are }\ControlFlowTok{in}\NormalTok{ the same working directory}
\NormalTok{    FILE IS Album Sales.csv; }\SpecialCharTok{!}\NormalTok{ Subcommands should end with ;}

    \SpecialCharTok{!}\NormalTok{ When data and input file are }\ControlFlowTok{in}\NormalTok{ the different working directory}
    \SpecialCharTok{!}\NormalTok{ FILE IS c}\SpecialCharTok{:}\NormalTok{\textbackslash{}desktop\textbackslash{}different folder\textbackslash{}Album Sales.csv;}

\SpecialCharTok{!}\NormalTok{ Variable command}
\NormalTok{VARIABLE}\SpecialCharTok{:}
    \SpecialCharTok{!}\NormalTok{ Column }\FunctionTok{names}\NormalTok{ (i.e., ALL variable names)}
\NormalTok{    NAMES ARE adverts sales airplay attract;}
    
    \SpecialCharTok{!}\NormalTok{ Variables that will be used }\ControlFlowTok{in}\NormalTok{ our analysis}
\NormalTok{    USEVARIABLES ARE adverts sales airplay;}
    
\SpecialCharTok{!}\NormalTok{ Analysis command}
\NormalTok{ANALYSIS}\SpecialCharTok{:}
\NormalTok{    ESTIMATOR IS ML; }\SpecialCharTok{!}\NormalTok{ This is the default}

\SpecialCharTok{!}\NormalTok{ Model command}
\NormalTok{MODEL}\SpecialCharTok{:}
    \SpecialCharTok{!}\NormalTok{ Let}\StringTok{\textquotesingle{}s predict sales using adverts and airplay}
\StringTok{    ! We regress sales on adverts and airplay}
\StringTok{    sales ON adverts airplay;}

\StringTok{    ! If you do not specify variances of and covariances between predictors}
\StringTok{    ! degrees of freedom (DF) are not correct}
\StringTok{    ! Variances of exogenous variable}
\StringTok{    adverts airplay;}
\StringTok{    ! Covariances between exogenous variable}
\StringTok{    adverts WITH airplay;}

\StringTok{! Output command}
\StringTok{OUTPUT:}
\StringTok{    TECH1 SAMPSTAT STDYX;}
\StringTok{    ! TECH1 to understand which parameters are being estimated}
\StringTok{    ! SAMPSTAT to check sample descriptive statistics}
\StringTok{    ! STDYX to standardize Y (i.e., DV) and X (i.e., IV)}
\end{Highlighting}
\end{Shaded}

\subsection{Part of the output file}\label{part-of-the-output-file-1}

\begin{Shaded}
\begin{Highlighting}[]
\NormalTok{MODEL RESULTS}

\NormalTok{                                                    Two}\SpecialCharTok{{-}}\NormalTok{Tailed}
\NormalTok{                    Estimate       S.E.  Est.}\SpecialCharTok{/}\NormalTok{S.E.    P}\SpecialCharTok{{-}}\NormalTok{Value}

\NormalTok{ SALES    ON}
\NormalTok{    ADVERTS            }\FloatTok{0.087}      \FloatTok{0.007}     \FloatTok{12.082}      \FloatTok{0.000}
\NormalTok{    AIRPLAY            }\FloatTok{3.589}      \FloatTok{0.285}     \FloatTok{12.608}      \FloatTok{0.000}

\NormalTok{ ADVERTS  WITH}
\NormalTok{    AIRPLAY          }\FloatTok{604.061}    \FloatTok{421.412}      \FloatTok{1.433}      \FloatTok{0.152}

\NormalTok{ Means}
\NormalTok{    ADVERTS          }\FloatTok{614.412}     \FloatTok{34.255}     \FloatTok{17.936}      \FloatTok{0.000}
\NormalTok{    AIRPLAY           }\FloatTok{27.500}      \FloatTok{0.865}     \FloatTok{31.777}      \FloatTok{0.000}
\end{Highlighting}
\end{Shaded}

\section{Additional materials}\label{additional-materials-1}

\begin{enumerate}
\def\labelenumi{\arabic{enumi}.}
\item
  Official website at \url{https://www.statmodel.com/}
\item
  User's guide and examples at \url{https://www.statmodel.com/ugexcerpts.shtml} \(\rightarrow\) Highly recommended!
\item
  Mplus YouTube channel at \url{https://www.youtube.com/c/MplusVideos}
\item
  QuantFish YouTube channel at \url{https://www.youtube.com/c/QuantFish}
\item
  Tutorials by Prof.~Rens van de Schoot and his students at \url{https://www.rensvandeschoot.com/tutorials/}
\end{enumerate}

\chapter{Summarizing Data}\label{summarizing-data}

\section{What is it? Why called Mplus?}\label{what-is-it-why-called-mplus-2}

Mplus is a statistical modeling program that provides researchers with a flexible tool to analyze data

\begin{itemize}
\item
  Many models: regression, path analysis, factor analysis, SEM, MLM, longitudinal models, mixture model, mediation/moderation
\item
  Many data: cross-sectional, longitudinal, single-/multilevel, observed/latent, incomplete
\item
  Many variables: continuous, dichotomous, categorical, count
\item
  Many estimator: maximum likelihood, weighted least squares, Bayesian
\end{itemize}

\section{Syntax-based programming}\label{syntax-based-programming-2}

\begin{itemize}
\item
  Commands and subcommands (\url{https://www.statmodel.com/language.html})
\item
  Examples of commands? (\url{https://www.youtube.com/watch?v=XeRRtdmu23k})

  \begin{itemize}
  \tightlist
  \item
    We will be `mostly' using TITLE, DATA, VARIABLE, ANALYSIS, MODEL, OUTPUT commands
  \item
    But we will also be often using DEFINE, SAVEDATA, PLOT, MONTECARLO commands
  \end{itemize}
\end{itemize}

\section{Some tips when programming}\label{some-tips-when-programming-2}

\begin{enumerate}
\def\labelenumi{\arabic{enumi}.}
\item
  Comments can be added with exclamation marks (!)
\item
  Commands should end with colon (:), and subcommands should end with semicolon (;)
\item
  Syntax is not case sensitive
\item
  Data should consist of numeric values, with no variable names
\item
  Data and Mplus input file should be in the same directory (like an R working directory)
\end{enumerate}

\begin{itemize}
\tightlist
\item
  Otherwise, be sure to specify the correct directory
\end{itemize}

\section{Some tips about model command particularly}\label{some-tips-about-model-command-particularly-2}

\begin{enumerate}
\def\labelenumi{\arabic{enumi}.}
\item
  Start with a path diagram
\item
  Think of it as specifying model parameters
\item
  Care to the degrees of freedom (DF)
\end{enumerate}

\section{Example: Multiple linear regression using maximum likelihood estimation}\label{example-multiple-linear-regression-using-maximum-likelihood-estimation-2}

\subsection{Model syntax}\label{model-syntax-2}

\begin{Shaded}
\begin{Highlighting}[]
\SpecialCharTok{!}\NormalTok{ Title command}
\NormalTok{TITLE}\SpecialCharTok{:}\NormalTok{ Predicting album sales using ML multiple regression}

\SpecialCharTok{!}\NormalTok{ Data command}
\NormalTok{DATA}\SpecialCharTok{:}
    \SpecialCharTok{!}\NormalTok{ When data and input file are }\ControlFlowTok{in}\NormalTok{ the same working directory}
\NormalTok{    FILE IS Album Sales.csv; }\SpecialCharTok{!}\NormalTok{ Subcommands should end with ;}

    \SpecialCharTok{!}\NormalTok{ When data and input file are }\ControlFlowTok{in}\NormalTok{ the different working directory}
    \SpecialCharTok{!}\NormalTok{ FILE IS c}\SpecialCharTok{:}\NormalTok{\textbackslash{}desktop\textbackslash{}different folder\textbackslash{}Album Sales.csv;}

\SpecialCharTok{!}\NormalTok{ Variable command}
\NormalTok{VARIABLE}\SpecialCharTok{:}
    \SpecialCharTok{!}\NormalTok{ Column }\FunctionTok{names}\NormalTok{ (i.e., ALL variable names)}
\NormalTok{    NAMES ARE adverts sales airplay attract;}
    
    \SpecialCharTok{!}\NormalTok{ Variables that will be used }\ControlFlowTok{in}\NormalTok{ our analysis}
\NormalTok{    USEVARIABLES ARE adverts sales airplay;}
    
\SpecialCharTok{!}\NormalTok{ Analysis command}
\NormalTok{ANALYSIS}\SpecialCharTok{:}
\NormalTok{    ESTIMATOR IS ML; }\SpecialCharTok{!}\NormalTok{ This is the default}

\SpecialCharTok{!}\NormalTok{ Model command}
\NormalTok{MODEL}\SpecialCharTok{:}
    \SpecialCharTok{!}\NormalTok{ Let}\StringTok{\textquotesingle{}s predict sales using adverts and airplay}
\StringTok{    ! We regress sales on adverts and airplay}
\StringTok{    sales ON adverts airplay;}

\StringTok{    ! If you do not specify variances of and covariances between predictors}
\StringTok{    ! degrees of freedom (DF) are not correct}
\StringTok{    ! Variances of exogenous variable}
\StringTok{    adverts airplay;}
\StringTok{    ! Covariances between exogenous variable}
\StringTok{    adverts WITH airplay;}

\StringTok{! Output command}
\StringTok{OUTPUT:}
\StringTok{    TECH1 SAMPSTAT STDYX;}
\StringTok{    ! TECH1 to understand which parameters are being estimated}
\StringTok{    ! SAMPSTAT to check sample descriptive statistics}
\StringTok{    ! STDYX to standardize Y (i.e., DV) and X (i.e., IV)}
\end{Highlighting}
\end{Shaded}

\subsection{Part of the output file}\label{part-of-the-output-file-2}

\begin{Shaded}
\begin{Highlighting}[]
\NormalTok{MODEL RESULTS}

\NormalTok{                                                    Two}\SpecialCharTok{{-}}\NormalTok{Tailed}
\NormalTok{                    Estimate       S.E.  Est.}\SpecialCharTok{/}\NormalTok{S.E.    P}\SpecialCharTok{{-}}\NormalTok{Value}

\NormalTok{ SALES    ON}
\NormalTok{    ADVERTS            }\FloatTok{0.087}      \FloatTok{0.007}     \FloatTok{12.082}      \FloatTok{0.000}
\NormalTok{    AIRPLAY            }\FloatTok{3.589}      \FloatTok{0.285}     \FloatTok{12.608}      \FloatTok{0.000}

\NormalTok{ ADVERTS  WITH}
\NormalTok{    AIRPLAY          }\FloatTok{604.061}    \FloatTok{421.412}      \FloatTok{1.433}      \FloatTok{0.152}

\NormalTok{ Means}
\NormalTok{    ADVERTS          }\FloatTok{614.412}     \FloatTok{34.255}     \FloatTok{17.936}      \FloatTok{0.000}
\NormalTok{    AIRPLAY           }\FloatTok{27.500}      \FloatTok{0.865}     \FloatTok{31.777}      \FloatTok{0.000}
\end{Highlighting}
\end{Shaded}

\section{Additional materials}\label{additional-materials-2}

\begin{enumerate}
\def\labelenumi{\arabic{enumi}.}
\item
  Official website at \url{https://www.statmodel.com/}
\item
  User's guide and examples at \url{https://www.statmodel.com/ugexcerpts.shtml} \(\rightarrow\) Highly recommended!
\item
  Mplus YouTube channel at \url{https://www.youtube.com/c/MplusVideos}
\item
  QuantFish YouTube channel at \url{https://www.youtube.com/c/QuantFish}
\item
  Tutorials by Prof.~Rens van de Schoot and his students at \url{https://www.rensvandeschoot.com/tutorials/}
\end{enumerate}

\chapter{Regression and ANOVA}\label{regression-and-anova}

\section{What is it? Why called Mplus?}\label{what-is-it-why-called-mplus-3}

Mplus is a statistical modeling program that provides researchers with a flexible tool to analyze data

\begin{itemize}
\item
  Many models: regression, path analysis, factor analysis, SEM, MLM, longitudinal models, mixture model, mediation/moderation
\item
  Many data: cross-sectional, longitudinal, single-/multilevel, observed/latent, incomplete
\item
  Many variables: continuous, dichotomous, categorical, count
\item
  Many estimator: maximum likelihood, weighted least squares, Bayesian
\end{itemize}

\section{Syntax-based programming}\label{syntax-based-programming-3}

\begin{itemize}
\item
  Commands and subcommands (\url{https://www.statmodel.com/language.html})
\item
  Examples of commands? (\url{https://www.youtube.com/watch?v=XeRRtdmu23k})

  \begin{itemize}
  \tightlist
  \item
    We will be `mostly' using TITLE, DATA, VARIABLE, ANALYSIS, MODEL, OUTPUT commands
  \item
    But we will also be often using DEFINE, SAVEDATA, PLOT, MONTECARLO commands
  \end{itemize}
\end{itemize}

\section{Some tips when programming}\label{some-tips-when-programming-3}

\begin{enumerate}
\def\labelenumi{\arabic{enumi}.}
\item
  Comments can be added with exclamation marks (!)
\item
  Commands should end with colon (:), and subcommands should end with semicolon (;)
\item
  Syntax is not case sensitive
\item
  Data should consist of numeric values, with no variable names
\item
  Data and Mplus input file should be in the same directory (like an R working directory)
\end{enumerate}

\begin{itemize}
\tightlist
\item
  Otherwise, be sure to specify the correct directory
\end{itemize}

\section{Some tips about model command particularly}\label{some-tips-about-model-command-particularly-3}

\begin{enumerate}
\def\labelenumi{\arabic{enumi}.}
\item
  Start with a path diagram
\item
  Think of it as specifying model parameters
\item
  Care to the degrees of freedom (DF)
\end{enumerate}

\section{Example: Multiple linear regression using maximum likelihood estimation}\label{example-multiple-linear-regression-using-maximum-likelihood-estimation-3}

\subsection{Model syntax}\label{model-syntax-3}

\begin{Shaded}
\begin{Highlighting}[]
\SpecialCharTok{!}\NormalTok{ Title command}
\NormalTok{TITLE}\SpecialCharTok{:}\NormalTok{ Predicting album sales using ML multiple regression}

\SpecialCharTok{!}\NormalTok{ Data command}
\NormalTok{DATA}\SpecialCharTok{:}
    \SpecialCharTok{!}\NormalTok{ When data and input file are }\ControlFlowTok{in}\NormalTok{ the same working directory}
\NormalTok{    FILE IS Album Sales.csv; }\SpecialCharTok{!}\NormalTok{ Subcommands should end with ;}

    \SpecialCharTok{!}\NormalTok{ When data and input file are }\ControlFlowTok{in}\NormalTok{ the different working directory}
    \SpecialCharTok{!}\NormalTok{ FILE IS c}\SpecialCharTok{:}\NormalTok{\textbackslash{}desktop\textbackslash{}different folder\textbackslash{}Album Sales.csv;}

\SpecialCharTok{!}\NormalTok{ Variable command}
\NormalTok{VARIABLE}\SpecialCharTok{:}
    \SpecialCharTok{!}\NormalTok{ Column }\FunctionTok{names}\NormalTok{ (i.e., ALL variable names)}
\NormalTok{    NAMES ARE adverts sales airplay attract;}
    
    \SpecialCharTok{!}\NormalTok{ Variables that will be used }\ControlFlowTok{in}\NormalTok{ our analysis}
\NormalTok{    USEVARIABLES ARE adverts sales airplay;}
    
\SpecialCharTok{!}\NormalTok{ Analysis command}
\NormalTok{ANALYSIS}\SpecialCharTok{:}
\NormalTok{    ESTIMATOR IS ML; }\SpecialCharTok{!}\NormalTok{ This is the default}

\SpecialCharTok{!}\NormalTok{ Model command}
\NormalTok{MODEL}\SpecialCharTok{:}
    \SpecialCharTok{!}\NormalTok{ Let}\StringTok{\textquotesingle{}s predict sales using adverts and airplay}
\StringTok{    ! We regress sales on adverts and airplay}
\StringTok{    sales ON adverts airplay;}

\StringTok{    ! If you do not specify variances of and covariances between predictors}
\StringTok{    ! degrees of freedom (DF) are not correct}
\StringTok{    ! Variances of exogenous variable}
\StringTok{    adverts airplay;}
\StringTok{    ! Covariances between exogenous variable}
\StringTok{    adverts WITH airplay;}

\StringTok{! Output command}
\StringTok{OUTPUT:}
\StringTok{    TECH1 SAMPSTAT STDYX;}
\StringTok{    ! TECH1 to understand which parameters are being estimated}
\StringTok{    ! SAMPSTAT to check sample descriptive statistics}
\StringTok{    ! STDYX to standardize Y (i.e., DV) and X (i.e., IV)}
\end{Highlighting}
\end{Shaded}

\subsection{Part of the output file}\label{part-of-the-output-file-3}

\begin{Shaded}
\begin{Highlighting}[]
\NormalTok{MODEL RESULTS}

\NormalTok{                                                    Two}\SpecialCharTok{{-}}\NormalTok{Tailed}
\NormalTok{                    Estimate       S.E.  Est.}\SpecialCharTok{/}\NormalTok{S.E.    P}\SpecialCharTok{{-}}\NormalTok{Value}

\NormalTok{ SALES    ON}
\NormalTok{    ADVERTS            }\FloatTok{0.087}      \FloatTok{0.007}     \FloatTok{12.082}      \FloatTok{0.000}
\NormalTok{    AIRPLAY            }\FloatTok{3.589}      \FloatTok{0.285}     \FloatTok{12.608}      \FloatTok{0.000}

\NormalTok{ ADVERTS  WITH}
\NormalTok{    AIRPLAY          }\FloatTok{604.061}    \FloatTok{421.412}      \FloatTok{1.433}      \FloatTok{0.152}

\NormalTok{ Means}
\NormalTok{    ADVERTS          }\FloatTok{614.412}     \FloatTok{34.255}     \FloatTok{17.936}      \FloatTok{0.000}
\NormalTok{    AIRPLAY           }\FloatTok{27.500}      \FloatTok{0.865}     \FloatTok{31.777}      \FloatTok{0.000}
\end{Highlighting}
\end{Shaded}

\section{Additional materials}\label{additional-materials-3}

\begin{enumerate}
\def\labelenumi{\arabic{enumi}.}
\item
  Official website at \url{https://www.statmodel.com/}
\item
  User's guide and examples at \url{https://www.statmodel.com/ugexcerpts.shtml} \(\rightarrow\) Highly recommended!
\item
  Mplus YouTube channel at \url{https://www.youtube.com/c/MplusVideos}
\item
  QuantFish YouTube channel at \url{https://www.youtube.com/c/QuantFish}
\item
  Tutorials by Prof.~Rens van de Schoot and his students at \url{https://www.rensvandeschoot.com/tutorials/}
\end{enumerate}

  \bibliography{book.bib,packages.bib}

\end{document}
